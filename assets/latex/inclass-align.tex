
\documentclass[11pt]{article}
%\usepackage{txfonts}
\usepackage{graphicx}
\usepackage{mygb4e}
\usepackage{solution}
\usepackage{hyperref}
\usepackage{amsmath}
\usepackage{tipa}

\setlength\oddsidemargin{0.01in}
\setlength\topmargin{-1in}
\setlength\textwidth{6.9in}
\setlength\textheight{9.5in} 

\newcommand{\nl}{\mbox{$\langle cr \rangle$}}

\raggedright
\raggedright
\newcommand{\blank}{\mbox{\verb*| |}}
\newcommand{\code}[1]{\texttt{\small #1}}

\newcommand{\argmax}{\operatornamewithlimits{argmax}}
\newcommand{\langa}{\textsc{Martian-A}}
\newcommand{\langb}{\textsc{Martian-B}}

\hidesolutions

\begin{document}

\begin{center}
{\Large Natural Language Processing}\\
{\large In-class Word Alignment Exercise}\\
{\small Anoop Sarkar}\\
\url{http://anoopsarkar.github.io/nlp-class} \\
This material is \copyright Anoop Sarkar 2014. \\
Only students registered for this course are allowed to download this material. \\
Use of this material for ``tutoring'' is prohibited.
\end{center}

\begin{exe}

\ex\label{kkex} \textbf{Human Translation} 

NASA's latest mission to Mars has found some strange
tablets. One tablet seems to be a kind of Rosetta stone which
has translations from a language we will call \langa\ 
(sentences 1a to 12a below)
to another language we will call \langb\ (sentences 1b to 12b below). 
The ASCII transcription of the alien script on the Rosetta tablet is given below:

\begin{center}
\begin{minipage}[t]{3.2in}
{\small\begin{verbatim}
1a. ok'sifar zvau hu .

1b. at'sifar somuds geyu .


2a. ok'anko ok'sifar myi pell hu .

2b. at'anko at'sifar ashi erder geyu .


3a. oprashyo hu qebb yuzvo oxloyzo .

3b. diza geyu isvat iwla pown .


4a. ok'sifar myi rig bzayr zu .

4b. at'sifar keerat ashi parq up .


5a. yux druh qebb stovokor .

5b. diza viodaws pai shun .


6a. ked hu qebb zu stovokor .

6b. dimbe geyu keerat pai shun .


7a. ked druh zvau ked hu qebb pnah .

7b. dimbe viodaws somuds dimbe geyu iwla woq .
\end{verbatim}
}
\end{minipage}
\begin{minipage}[t]{3in}
{\small\begin{verbatim}

8a. ked bzayr myi pell eoq .

8b. gakh up ashi erder kvig .


9a. yux eoq qebb zada ok'nefos .

9b. diza kvig pai goli at'nefos .


10a. ked amn eoq kin oxloyzo hom .

10b. dimbe kvig baz iluh ejuo pown .


11a. ked eoq tazih yuzvo kin dabal'ok .

11b. dimbe kvig isvat iluh dabal'at .


12a. ked mina eoq qebb yuzvo amn .

12b. dimbe kvig zeg isvat iwla baz .
\end{verbatim}
}
\end{minipage}
\end{center}

We would like to create a translation from the source language
which we will take to be \langb\ and produce output in the target
language which will be \langa. Due to severe budget cutbacks at NASA, decryption of these
tablets has fallen to people like you. 
In this question, you should try to solve this task
by hand to get some insight into the process of translation.

\begin{xlist}

\ex{\label{kkex1} Use the above translations to
produce a translation dictionary. For each word in \langa\
provide an equivalent word in \langb. If a word in \langa\ has 
no equivalent in \langb\ then put the entry ``(none)'' in the dictionary.

\begin{soln}
\begin{center}
\begin{tabular}{|l|l|}
\hline
\langa & \langb \\
\hline
myi & ashi \\
bzayr & up \\
hom & ejuo \\
tazih & (none) \\
rig & parq \\
pnah & woq \\
oprashyo & diza \\
druh & viodaws \\
oxloyzo & pown \\
yuzvo & isvat \\
qebb & iwla \\
qebb & pai \\
zu & keerat \\
zada & goli \\
ked & dimbe \\
ked & gakh \\
amn & baz \\
eoq & kvig \\
ok'anko & at'anko \\
ok'sifar & at'sifar \\
ok'nefos & at'nefos \\
zvau & somuds \\
pell & erder \\
mina & zeg \\
hu & geyu \\
stovokor & shun \\
yux & diza \\
kin & iluh \\
dabal'ok & dabal'at \\
\hline 
\end{tabular}
\end{center}
\end{soln}

}

\ex{\label{kkex2}  Using your translation dictionary, provide a word for word
translation for the following \langb\ sentences on a new tablet which was found
near the Rosetta tablet. 
{\small\begin{verbatim}
13b. gakh up ashi woq pown goli at'nefos .

14b. diza kvig zeg isvat iluh ejuo .

15b. dimbe geyu pai shun hunslob at'anko .
\end{verbatim}
}

The \langa\ sentences you
produce will probably appear to be in a different word order from
the \langa\ sentences you observed on the Rosetta tablet. Some
words might be unseen and so seemingly untranslatable. In those
cases insert the word {\small\tt ?} for the unseen word.

\begin{soln}
{\small\begin{verbatim}
13a. ked bzayr myi pnah oxloyzo zada ok'nefos .

14a. yux eoq mina yuzvo kin hom .

15a. ked hu qebb stovokor ? ok'anko .
\end{verbatim}
}
\end{soln}

}

\ex{\label{kkex3}  The word for word translation can be improved with
additional knowledge about \langa\ word order. Luckily another
tablet containing some \langa\ sentences (untranslated) was
found on the dusty plains of Mars. Use these \langa\ sentences
in order to find the most plausible word order for the \langa\ sentences
translated from \langb\ sentences in (\ref{kkex2}).

{\small\begin{verbatim}
ok'anko myi oxloyzo druh .
yux mina eoq esky oxloyzo pnah .
ok'anko yolk stovokor koos oprashyo pnah zada ok'nefos yun zu kin hom .
ked hom qebb koos ok'anko .
ok'sifar zvau hu .
ok'anko ok'sifar
myi pell hu .
oprashyo hu qebb yuzvo oxloyzo .
ok'sifar myi rig bzayr zu .
yux druh qebb stovokor .
ked hu qebb zu stovokor .
ked bzayr myi pell eoq .
ked druh zvau ked hu qebb pnah .
yux eoq qebb zada ok'nefos .
ked amn eoq kin oxloyzo hom .
ked eoq tazih yuzvo kin dabal'ok .
ked mina eoq qebb yuzvo amn .
\end{verbatim}
}

Using this additional \langa\ text you can even find a translation for words that 
are missing from the translation dictionary (although this might be hard to implement
in a program, cases that were previously translated as {\small\tt ?} can be 
translated by manual inspection of the above \langa\ text).

\begin{soln}
{\small\begin{verbatim}
13a. ked bzayr myi oxloyzo pnah zada ok'nefos .

14a. yux mina eoq tazih yuzvo kin hom .

15a. ked hu qebb stovokor koos ok'anko .
\end{verbatim}
}
\end{soln}
}

\end{xlist}

\newpage
\ex The following is a small parallel text (the same text in two different
languages). The 1st column contains phrases in Udihe. The 2nd column contains the
English equivalent. 

\bigskip

\begin{tabular}{ll}
\textipa{b"ata z\"a:Nini} & the boy's money \\
\textipa{si bogdoloi} & thy shoulder \\
\textipa{ja: xabani} & the cow's udder \\
\textipa{su z\"a:Niu} & your money \\
\textipa{dili tekpuni} & the skin of the head \\
\textipa{si ja:Ni:} & thy cow \\
\textipa{bi mo:Ni:} & my tree \\
\textipa{aziga bugdini} & the girl's leg \\
\textipa{bi nakta diliNi:} & my boar head \\
\textipa{nakta igini} & the boar's tail \\
\textipa{si b"ataNi: bogdoloni} & thy son's shoulder \\
\textipa{teNku bugdini} & the leg of the stool \\
\textipa{su ja: wo:Niu} & your cow thigh \\
\textipa{bi wo:i} & my thigh
\end{tabular}

\bigskip

\textipa{N}, \textipa{"} are consonants, \textipa{\"a} is a vowel. The
\textipa{:} indicates length of preceding vowel (so for example \textipa{i}+\textipa{i} is written as \textipa{i:}). 
The archaic English {\it thy} is used to indicate singular and {\it your} is used to indicate plural.

\bigskip

\begin{soln}

\smallskip

Consider the English phrase {\it X's Y} or {\it Y of the X}. The following table
summarizes how this phrase has to be structured in Udihe:

\smallskip

\begin{tabular}{|l|p{1.4in}|p{3in}|}
\hline
X (possessor) & Udihe phrase for {\it X's Y} or {\it Y of the X} & examples \\
\hline
singular \& I/you/my/thy & X Y-(\textipa{Ni})-\textipa{i} & 
  \textipa{bi wo:\underline{i}}, \textipa{bi mo:\underline{Ni:}},
  \textipa{bi nakta dili\underline{Ni:}},
  \textipa{si bogdolo\underline{i}}, \textipa{si ja:\underline{Ni:}},
   \textipa{si b"ata\underline{Ni:} bogdoloni} \\
singular (all other cases) & X Y-(\textipa{Ni})-\textipa{ni} &
  \textipa{ja: xaba\underline{ni}}, \textipa{dili tekpu\underline{ni}}, 
  \textipa{b"ata z\"a:\underline{Nini}}, 
  \textipa{si b"ataNi: bogdolo\underline{ni}} \\
plural & X Y-(\textipa{Ni})-\textipa{u} & 
  \textipa{su z\"a:\underline{Niu}}, \textipa{su ja: wo:\underline{Niu}} \\
\hline
\end{tabular}

\smallskip

Notice that \textipa{Ni} occurs exactly in those cases when, in the phrase
{\it X's Y}, the {\it Y} is not in a part-whole relationship with respect
to {\it X}. For example, \textipa{bi wo:i} (my thigh) is in a part-whole
relationship, while \textipa{bi mo:Ni:} (my tree) is not in a part-whole
relationship. Also, \textipa{Ni}+\textipa{i} becomes \textipa{Ni:} since
the vowel is simply lengthened.

\smallskip

In the case where the possessor is itself a possession phrase
e.g. {\it \fbox{thy son}'s shoulder}, each possessee gets the 
appropriate suffix, e.g. \textipa{si b"ata\underline{Ni:} bogdolo\underline{ni}}.
And in the case where the possessee is itself a possession phrase, 
e.g.  {\it my \fbox{boar head}}, only the actual part possessed is
marked, e.g. \textipa{bi nakta dili\underline{Ni:}}. How would you say
{\it the skin of the head of the cow} in Udihe?

\bigskip

Consider the pronouns observed in the parallel text:

\smallskip

\begin{tabular}{|c|c|}
\hline
singular & plural \\
\hline
\textipa{bi}/I & (\textipa{bu})/our \\
\textipa{si}/you & \textipa{su}/your \\
\hline
\end{tabular}

\smallskip

Note that the pronoun {\it our} does not occur in the text, but by
analogy to {\it you} vs. {\it your} we can conjecture that the plural
of {\it I} which is {\it our} in English, will be \textipa{bu} in Udihe.

\bigskip

Another missing form we can construct using analogy is the word for
{\it daughter} which is not observed, but we do observe the words for
{\it boy} and {\it son}:

\smallskip

\begin{tabular}{|c|c|}
\hline
\textipa{b"ata}/boy & \textipa{b"ata}/son \\
\textipa{aziga}/girl & (\textipa{aziga})/daughter \\
\hline
\end{tabular}

\end{soln}

\ex Translate into English:

\begin{xlist}
{\ex \textipa{su b"ataNiu z\"a:Nini}

\begin{soln}
your son's money
\end{soln}
}

{\ex \textipa{si teNku bugdiNi:}

\begin{soln}
thy stool leg
\end{soln}
}

{\ex 
\textipa{si teNkuNi: bugdini}

\begin{soln}
thy stool's leg
\end{soln}
}

\end{xlist}

\ex Translate into Udihe:

\begin{xlist}
{\ex the boy's thigh

\begin{soln}
\textipa{b"ata wo:ni}
\end{soln}
}

{\ex our boar

\begin{soln}
\textipa{bu naktaNiu}
\end{soln}
}

{\ex my daughter's tree

\begin{soln}
\textipa{bi azigaNi: mo:Nini}
\end{soln}
}

\end{xlist}

Udihe speakers mostly live in the Siberian far east, 
and the language is classified as belonging to the Tungus-Manchu language
family. There are roughly 100 people who still speak this language. The language
is almost extinct. Other than the parallel text given above, you do not need any knowledge about the language
and its speakers to answer the questions, but if you are curious, here are
some web pages on the Udihe language:

{\small
\begin{verbatim}
http://www.ethnologue.com/show_language.asp?code=ude
http://en.wikipedia.org/wiki/Udege_language
\end{verbatim}
}

\begin{comment}
{\bf Hint}
\bigskip

All the expressions above denote something (call it ${\cal Y}$) belonging to someone
or something else (call it ${\cal X}$). In English, the predicate of possession
$\textsf{Possess}({\cal X}, {\cal Y})$
would be denoted by the phrase {\it the ${\cal Y}$ of the ${\cal X}$} / {\it the ${\cal X}$'s ${\cal Y}$}, 
e.g. {\it the stool's leg} 
or {\it the boy's money}. How is this predicate expressed
in Udihe? 

\bigskip

In English, we can use the expression ${\cal X}$'s ${\cal Y}$ 
if the predicate indicates that ${\cal Y}$ is a part of ${\cal X}$, 
e.g. {\it the stool's leg}, or if the predicate indicates that ${\cal Y}$ is separate from 
and belongs to ${\cal X}$, e.g. {\it the boy's money}. How are these two predicates
expressed in Udihe? 

\bigskip

Things get a bit more interesting when we add in another component and we have
two possession predicates. Depending on how they interact, we can observe
different cases: for instance,  
${\cal X}$ ${\cal Y}$'s ${\cal Z}$, e.g. {\it \fbox{thy son}'s shoulder}  or
${\cal X}$'s ${\cal Y}$ ${\cal Z}$, e.g. {\it my \fbox{boar head}}.

\bigskip
\bigskip
\end{comment}

{\small
Thanks to B. Iomdin who originally created the parallel text and the concept
behind the question for an international olympiad in computational linguistics.
The question has been somewhat simplified to make the computational aspect of
the translation more explicit.
}


\end{exe}

\end{document}

