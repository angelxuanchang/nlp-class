
\documentclass[12pt]{article}
\usepackage{txfonts}
\usepackage{graphicx}
\usepackage{mygb4e}
\usepackage{solution}

\hidesolutions
\setlength\oddsidemargin{0.01in}
\setlength\topmargin{-1in}
\setlength\textwidth{6.9in}
\setlength\textheight{9.5in}

\newcommand{\nl}{\mbox{$\langle cr \rangle$}}
\newcommand{\decaf}{\mbox{\textbf{\texttt{Decaf}}}}
\newcommand{\kw}[1]{\textbf{#1}}
\newcommand{\ident}[1]{\texttt{#1}}
\newcommand{\nonterm}[1]{\mbox{$\langle\mbox{#1}\rangle$\ }}
\newcommand{\term}[1]{\mbox{\textbf{#1}\ }}
\newcommand{\termchar}[1]{\mbox{\textbf{`\texttt{#1}'}\ }}
\newcommand{\once}[1]{\mbox{$\Big[\mbox{#1}\Big]$\ }}
\newcommand{\kstar}[1]{\mbox{\mbox{#1}$^\ast$\ }}
\newcommand{\plus}[1]{\mbox{\mbox{#1}$^+$\ }}
\newcommand{\leftgroup}{\mbox{\textbf{\Big\{}\ }}
\newcommand{\rightgroup}{\mbox{\textbf{\Big\}\ }}}
\newcommand{\group}[1]{\mbox{\leftgroup\ #1\ \rightgroup}\ }
\newcommand{\charconst}[1]{'\texttt{#1}'\ }
\newcommand{\sep}{\mbox{$\mid$}\ \ }
\newcommand{\commasep}[1]{\mbox{\leftgroup\mbox{#1}\rightgroup$^+$\textbf{,}}\ }
\newcommand{\cfgrule}[2]{#1 & \rightarrow & #2 \nonumber}
\newcommand{\nlcfgrule}[2]{\textrm{#1} & \rightarrow & \textrm{#2} \nonumber}

\raggedright

\newcommand{\valueof}[1]{\lbrack\!\lbrack #1 \rbrack\!\rbrack}
\newcommand{\refp}[1]{(\protect\ref{#1})}
\def\rwd#1 {\mbox{#1}}
\def\N{\ensuremath{\mathcal{N}}}
\def\implies{\ensuremath{\Rightarrow}}
% Set some text inside an fbox the full width of the line, with the frame
% sticking out into the margin.
\long\def\framepar#1{\par\noindent\hbox to \textwidth {\hskip-\fboxsep
\fbox{\parbox{\the\textwidth}{#1}}}}

\begin{document}
%\setlength{\baselineskip}{12pt}

\begin{center}
{\Large\bf
Context-free Grammars: In-class Exercise}\\
\end{center}

\bigskip

\begin{exe}


\ex\label{tree} Consider the CFG $G$ with $S'$ as the start symbol:
\begin{eqnarray}
S' & \rightarrow & S \mid \epsilon \nonumber\\
S  & \rightarrow & T \mid (\ N\ ,\ C\ ) \nonumber\\
C  & \rightarrow & C\ ,\ S \mid S \nonumber\\
T  & \rightarrow & a \mid b \mid c \nonumber\\
N  & \rightarrow & x \mid y \mid z \nonumber
\end{eqnarray}

\begin{xlist}

\ex{\label{tree1} List the set of terminal symbols and the set of non-terminal symbols in $G$.

\begin{soln}
\begin{eqnarray*}
T &=& \{ a, b, c, x, y, z, \backslash, , (, ) \} \\
N &=& \{ S', S, C, T, N \}
\end{eqnarray*}
\end{soln}

}

\ex{\label{tree2} For each of the following strings, write down {\tt true} if the string is in the language $L(G)$ generated by $G$, {\tt false} otherwise.
\begin{enumerate}
\item {\tt y} % false
\item {\tt c} % true
\item {\tt (x)} % false
\item {\tt (x,y)} % false
\item {\tt (z,a,b,a,b,c)} % true
\item {\tt (x,a,(y,b),c)} % true
\item {\tt (x,(y,a),(z,b))} % true
\item {\tt (x,(x,(x,(x,a))} % false
\end{enumerate}

\begin{soln}
\begin{enumerate}
\item {\tt y} : false
\item {\tt c} : true
\item {\tt (x)} : false
\item {\tt (x,y)} : false
\item {\tt (z,a,b,a,b,c)} : true
\item {\tt (x,a,(y,b),c)} : true
\item {\tt (x,(y,a),(z,b))} : true
\item {\tt (x,(x,(x,(x,a))} : false
\end{enumerate}
\end{soln}

}

\begin{comment}
\ex{\label{tree3} Write down the leftmost derivation for the string {\tt (x,(y,b))}

\begin{soln}
\begin{eqnarray*}
S' & \Rightarrow & S \Rightarrow (N,C) \Rightarrow (x,C) \Rightarrow (x,S) \Rightarrow (x,(N,C)) \Rightarrow (x,(y,C)) \Rightarrow (x,(y,S)) \\
   & \Rightarrow & (x,(y,T)) \Rightarrow (x,(y,b))
\end{eqnarray*}
\end{soln}
}

\ex{\label{tree4} Write down the rightmost derivation for the string {\tt (x,(y,b))}

\begin{soln}
\begin{eqnarray*}
S' & \Rightarrow & S \Rightarrow (N,C) \Rightarrow (N,S) \Rightarrow (N,(N,C)) \Rightarrow (N,(N,S)) \Rightarrow (N,(N,T)) \\
   & \Rightarrow & (N,(N,b)) \Rightarrow (N,(y,b)) \Rightarrow (x,(y,b))
\end{eqnarray*}
\end{soln}

}

 \ex{\label{tree5} Draw the parse tree produced by $G$ for the string {\tt (x,(y,b))}

\begin{soln}
\begin{verbatim}
 (S' (S \(
        (N x)
        ,
        (C (S \(
              (N y)
              ,
              (C (S (T b)))
              \)))
        \)))
\end{verbatim}

\noindent Notice that the tree constructed using the leftmost derivation is identical
to the tree constructed using the rightmost derivation
\end{soln}

 }
\end{comment}

\end{xlist}

\bigskip
\ex\label{nlg}

One of the rules in the CFG below is redundant: any sentence that
can be generated using this rule can already be generated by a
combination of other rules. Write down the redundant rule.

\begin{minipage}[t]{5cm}
\begin{eqnarray*}
\nlcfgrule{S}{NP\ VP}\\
\nlcfgrule{NP}{N}\\
\nlcfgrule{NP}{D\ N}\\
\nlcfgrule{VP}{VP\ PP}\\
\nlcfgrule{VP}{VP\ CONJ\ VP}\\
\nlcfgrule{VP}{IV}\\
\nlcfgrule{VP}{IV\ PP}\\
\nlcfgrule{VP}{TV\ NP}\\
\nlcfgrule{VP}{TV\ C\ S}\\
\nlcfgrule{NP}{NP\ CONJ\ NP}\\
\nlcfgrule{PP}{P}\\
\nlcfgrule{PP}{P\ NP}\\
\end{eqnarray*}
\end{minipage}
\begin{minipage}[t]{5cm}
\begin{eqnarray*}
\nlcfgrule{IV}{runs}\\
\nlcfgrule{IV}{sits}\\
\nlcfgrule{TV}{chases}\\
\nlcfgrule{TV}{eats}\\
\nlcfgrule{TV}{catches}\\
\nlcfgrule{TV}{tells}\\
\nlcfgrule{TV}{sees}\\
\nlcfgrule{CONJ}{and}\\
\nlcfgrule{C}{that}\\
\nlcfgrule{P}{in}\\
\nlcfgrule{P}{away}\\
\end{eqnarray*}
\end{minipage}
\begin{minipage}[t]{5cm}
\begin{eqnarray*}
\nlcfgrule{N}{John}\\
\nlcfgrule{N}{he}\\
\nlcfgrule{N}{Mary}\\
\nlcfgrule{N}{dog}\\
\nlcfgrule{N}{tree}\\
\nlcfgrule{N}{squirrel}\\
\nlcfgrule{D}{the}\\
\end{eqnarray*}
\end{minipage}

\begin{soln}
$VP \rightarrow IV\ PP$ can be generated using $VP \rightarrow IV$ and $VP \rightarrow V\ PP$.
\end{soln}

\bigskip
\ex\label{derivs} Consider the family of CFGs $G_k$ with $S$ as the start symbol and $k$ is some arbitrary non-zero positive integer such that $G_1, G_2, G_3, \ldots$ are individual CFGs with the rules:
\begin{eqnarray*}
\cfgrule{S}{A\ B} \\
\cfgrule{B}{C\ A\ A} \\
\cfgrule{C}{c} \\
\cfgrule{A}{a_i \textrm{\ \ \ defines $i$ rules, where $i \in [1,k]$ }}
\end{eqnarray*}
For example, in $G_3$ the rules with $A$ as left-hand side are: $A \rightarrow a_1 \mid a_2 \mid a_3$ with three terminal symbols.

\begin{xlist}

{\ex Provide the number of terminal symbols in a grammar $G_k$.
\begin{soln}
$k+1$
\end{soln}
}

{\ex If the string $a_4 c a_3 a_2$ is accepted by grammar $G_3$ then provide a derivation for it.
\begin{soln}
$a_4$ does not exist as a terminal in $G_3$.
\end{soln}
}

{\ex If the string $a_4 c a_3 a_2$ is accepted by grammar $G_4$ then provide a derivation for it.
\begin{soln}
$S \Rightarrow A\ B \Rightarrow a_3\ B \Rightarrow a_3\ C\ A\ A \Rightarrow a_3\ c\ A\ A \Rightarrow a_3\ c\ a_1\ A \Rightarrow a_3\ c\ a_1\ a_2$
\end{soln}
}

{\ex Provide the total number of strings that can be generated for a grammar $G_k$.
\begin{soln}
$k^3$
\end{soln}
}

\end{xlist}

\bigskip
\ex\label{pcfgtree} Consider a treebank which consists of three tree {\em types}: $T_1, T_2, T_3$. In this treebank these tree types are repeated multiple times. By counting the number of times each tree type was observed, we discover that each tree type occurs with the following probability:

\begin{center}
\begin{tabular}{ll}
$p_1$ & $T_1 = $ (S (B a) (C a a)) \\
$p_2$ & $T_2 = $ (S (B a a)) \\
$p_3$ & $T_3 = $ (S (C a a a)) 
\end{tabular}
\end{center}

\begin{xlist}
\smallskip
\ex From the treebank shown above, extract a probabilistic CFG (PCFG) $G$. \\
Assume that the rule S $\rightarrow$ BC appears in $G$ with probability $p_1$ and $p_1 + p_2 + p_3 = 1$.

\begin{soln}
\begin{minipage}[t]{2.5in}
 \begin{tabular}{ll}
 $p_1$ & S $\rightarrow$ BC \\
 $p_2$ & S $\rightarrow$ B \\
 $p_3$ & S $\rightarrow$ C 
 \end{tabular}
\end{minipage}
\begin{minipage}[t]{2.5in}
 \begin{tabular}{ll}
 $\frac{p_1}{p_1 + p_2}$ & B $\rightarrow$ a \\
 $\frac{p_2}{p_1 + p_2}$ & B $\rightarrow$ aa \\
 $\frac{p_1}{p_1 + p_3}$ & C $\rightarrow$ aa \\
 $\frac{p_3}{p_1 + p_3}$ & C $\rightarrow$ aaa
 \end{tabular}
\end{minipage}
\end{soln}

\smallskip
\ex Provide the tree set ${\cal T}$ for the CFG $G$.

\begin{soln}

\begin{center}
\begin{minipage}[t]{2.5in}
 \begin{tabular}{ll}
$T_4 = $ & (S (B a) (C a a a)) \\
$T_5 = $ & (S (B a a) (C a a)) \\
$T_6 = $ & (S (B a a) (C a a a))
 \end{tabular}
\end{minipage}
\begin{minipage}[t]{2.5in}
 \begin{tabular}{ll}
$T_7 = $ & (S (B a)) \\
$T_8 = $ & (S (C a a))
 \end{tabular}
\end{minipage}
\end{center}
\[ {\cal T} = \{ T_1, T_2, T_3, T_4, T_5, T_6, T_7, T_8  \} \]
\end{soln}

\smallskip
\ex Provide the language ${\cal L}$ (the set of strings) for the CFG $G$.

\begin{soln}
\[ {\cal L} = \{ aaa, aa, aaaa, aaaaa, a \} \]
\end{soln}

\smallskip
\ex Let $p_1 = 0.2, p_2 = 0.1, p_3 = 0.7$. Find the parse tree with highest probability according to PCFG $G$ for the input string $aa$. Note that tree $T_2$ in the treebank is a tree that has yield $aa$. Write down if the tree you find is the same as $T_2$.

\begin{soln}
\begin{minipage}[t]{2.5in}
 \begin{tabular}{ll}
 $0.2$ & S $\rightarrow$ BC \\
 $0.1$ & S $\rightarrow$ B \\
 $0.7$ & S $\rightarrow$ C
 \end{tabular}
\end{minipage}
\begin{minipage}[t]{2.5in}
 \begin{tabular}{ll}
 $\frac{2}{3}$ & B $\rightarrow$ a \\
 $\frac{1}{3}$ & B $\rightarrow$ aa \\
 $\frac{2}{9}$ & C $\rightarrow$ aa \\
 $\frac{7}{9}$ & C $\rightarrow$ aaa
 \end{tabular}
\end{minipage}

$P(T_2) = 0.0333$ and $P(T_8) = 0.1555$, so $T_8$ is the parse tree with highest probability for input string $aa$.
\end{soln}
\end{xlist}

\end{exe}
\end{document}

